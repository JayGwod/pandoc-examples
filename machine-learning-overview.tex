% Options for packages loaded elsewhere
\PassOptionsToPackage{unicode}{hyperref}
\PassOptionsToPackage{hyphens}{url}
%
\documentclass[
]{extarticle}
\usepackage{lmodern}
\usepackage{amssymb,amsmath}
\usepackage{ifxetex,ifluatex}
\ifnum 0\ifxetex 1\fi\ifluatex 1\fi=0 % if pdftex
  \usepackage[T1]{fontenc}
  \usepackage[utf8]{inputenc}
  \usepackage{textcomp} % provide euro and other symbols
\else % if luatex or xetex
  \usepackage{unicode-math}
  \defaultfontfeatures{Scale=MatchLowercase}
  \defaultfontfeatures[\rmfamily]{Ligatures=TeX,Scale=1}
\fi
% Use upquote if available, for straight quotes in verbatim environments
\IfFileExists{upquote.sty}{\usepackage{upquote}}{}
\IfFileExists{microtype.sty}{% use microtype if available
  \usepackage[]{microtype}
  \UseMicrotypeSet[protrusion]{basicmath} % disable protrusion for tt fonts
}{}
\makeatletter
\@ifundefined{KOMAClassName}{% if non-KOMA class
  \IfFileExists{parskip.sty}{%
    \usepackage{parskip}
  }{% else
    \setlength{\parindent}{0pt}
    \setlength{\parskip}{6pt plus 2pt minus 1pt}}
}{% if KOMA class
  \KOMAoptions{parskip=half}}
\makeatother
\usepackage{xcolor}
\IfFileExists{xurl.sty}{\usepackage{xurl}}{} % add URL line breaks if available
\IfFileExists{bookmark.sty}{\usepackage{bookmark}}{\usepackage{hyperref}}
\hypersetup{
  pdftitle={A Review of Machine Learning},
  pdfauthor={Guo Dejie},
  pdfkeywords={test, test1, test2},
  hidelinks,
  pdfcreator={LaTeX via pandoc}}
\urlstyle{same} % disable monospaced font for URLs
\setlength{\emergencystretch}{3em} % prevent overfull lines
\providecommand{\tightlist}{%
  \setlength{\itemsep}{0pt}\setlength{\parskip}{0pt}}
\setcounter{secnumdepth}{5}

\title{A Review of Machine Learning}
\author{Guo Dejie}
\date{\today{}}

\begin{document}
\maketitle
\begin{abstract}
This review gives a big picture about machine learning.
\end{abstract}
\keywords{test}

\hypertarget{introduction}{%
\section{Introduction}\label{introduction}}

This overview provides a high level perspective of machine learning.
Firstly, the concept of machine learning would be discussed and then to
consider which necessary components should be included to build a
relatively complete learning system. ``Learning algorithm'' plays a
central role in the field of machine learning, therefore in the next
part the main ideas would be introduced: the framework of empirical risk
minimization, the principle of maximum likelihood and the idea of
probabilistic models. Besides algorithms, how to apply the techniques of
machine learning is another very important topic. In the last part, the
current best practices would be briefly outlined, including how to train
a model such that the resulting predictor does well on data not yet
seen, how to select a model and how to set up your machine learning
workflow.

\hypertarget{part-one}{%
\section{part one}\label{part-one}}

\end{document}
